\documentclass{vegaarticle}

\usepackage{import}
\addbibresource{refs.bib}

\author{Vsevolod Zaostrovsky, Ivan Cherepakhin, Artemy Sazonov}
\title{Forecasting the Yield Curve: An Econometric Study}
\date{\today}
\nocite{*}
\graphicspath{{fig/}}
\jel{C53, G17, E47}

\begin{document}
    \maketitle

    \begin{abstract}{}
        This research study employs econometric analysis techniques to investigate the forecasting of the yield curve,
        analyze impulse response functions (IRFs), and detect structural breaks. Accurate forecasting of the yield curve
        is crucial for investors, policymakers, and risk managers in making informed decisions. The analysis of IRFs
        provides insights into the dynamic response of the yield curve to shocks in macroeconomic variables, allowing
        for a deeper understanding of the transmission mechanisms. Additionally, the study examines structural breaks in
        the yield curve associated with unpredictable events, providing valuable insights into shifts in market
        dynamics. By combining these three components, this research contributes to a broader understanding of the yield
        curve's behavior and its implications for financial markets and economic policies.
    \end{abstract}

    \introduction
        The yield curve, depicting the relationship between time to maturity and yields on zero-coupon bonds, serves as
        a vital indicator of market expectations, economic conditions, and future monetary policy. Accurate forecasting
        of the yield curve has tremendous implications for investors, policymakers, and risk managers. Simultaneously,
        understanding the dynamic response of the yield curve to shocks and structural breaks enables a deeper
        comprehension of the underlying economic factors and their impact on financial markets.

        This paper aims to contribute to the existing literature by conducting a comprehensive econometric analysis that
        encompasses yield curve forecasting, impulse response analysis, and the investigation of structural breaks. By
        incorporating these three key components, this research seeks to shed light on the interplay between economic
        variables, forecast future yield curve movements, and detect shifts in the yield curve structure associated with
        unpredictable events.
        
        The first component of this study focuses on yield curve forecasting using econometric techniques, such as
        Vector Autoregression (VAR) models or Dynamic Nelson-Siegel (DNS) models. By leveraging historical data on
        zero-coupon yields and potential explanatory variables, the chosen model will provide forecasts for the yield
        curve over a specified time horizon. The accuracy of these forecasts will be rigorously evaluated using
        statistical measures such as mean absolute error (MAE) or root mean squared error (RMSE), allowing for an
        assessment of the model's predictive capabilities.
        
        In addition to forecasting, this research incorporates the analysis of impulse response functions (IRFs).
        Through estimating the dynamic response of the yield curve to shocks in relevant macroeconomic variables such as
        GDP growth, inflation, or monetary policy indicators, the IRFs provide insight into the transmission channels
        and the lagged effects of these shocks on the yield curve. This analysis will enhance our understanding of the
        interactions between the yield curve and important economic factors, contributing to the wider field of monetary
        policy and financial markets.
        
        Furthermore, we address the critical aspect of detecting and studying structural breaks in the yield curve.
        Unforeseen events, whether political, social, or economic in nature, can lead to significant shifts in the yield
        curve's structure. By employing robust econometric techniques, such as Chow tests, Bai-Perron tests, or
        Markov-switching models, this study will identify and examine these structural breaks. The timing, magnitude,
        and nature of the breaks will be analyzed, providing valuable insights into the factors driving the shifts and
        their implications for market dynamics.
        
        In summary, this research aims to offer a comprehensive analysis of the yield curve, incorporating yield curve
        forecasting, impulse response analysis, and the study of structural breaks. By combining these three aspects,
        we provide insights into the future movements of the yield curve, the dynamic response to macroeconomic shocks,
        and the detection of structural changes associated with unpredictable events. These findings hold significant
        implications for investors, policymakers, and market participants, ultimately contributing to a deeper
        understanding of macroeconomic dynamics and aiding in informed decision-making in financial markets.

    %\import{./}{Literature_review.tex}

    \import{./}{Data.tex}

    \import{./}{YC_Forecasting.tex}

    \import{./}{IRA.tex}

    \import{./}{SBA.tex}

    \section{Conclusion}
        \par
        This research leads one to three important conclusion:
        \begin{itemize}
            \item Now it is clear why classical time series without exogenous parameters are not used to predict bond yields. 
            The models are only capable of offering constant or constant-like predictions, and the bond yields are considered 
            martingales within the framework of these models. In order to obtain more meaningful predictions, additional 
            information about the bond is necessary, such as specific and strong model assumptions or a suitable set of exogenous 
            factors. Without such information, the best outcome we can achieve is a constant-like prediction. This distinction 
            further emphasizes the lack of memory in bonds, unlike stocks, which can exhibit more complex and dynamic behavior over time.
            \item In our research, we have made a significant discovery that bonds with similar times to maturity are cointegrated. 
            This finding is particularly noteworthy and can be explained by the theory of market segmentation. 
            Bonds with different maturities tend to attract different categories of investors, who have largely 
            dissimilar preferences and interests. This distinct segmentation contributes to the formation of a specific relationship 
            between bond markets with close time to maturity. Furthermore, this discovery holds great importance from a technical perspective,
             as it affects the methodology of predictions and the construction of impulse responses.
            \item Unfortunately, despite our efforts, we were unable to construct non constant-like forecasts for the coefficients 
            in the considered version of the Nelson-Siegel model. It is possible that these coefficients are also martingales under the 
            filtration used in the Russian market, suggesting the need for a more intricate model to derive non-trivial predictions. 
            Additionally, this could be attributed to the specific period we analyzed, a post-crisis growth period characterized 
            by exceptional conditions in the Russian economy. During this time, the economy was free from significant economic shocks, 
            and there was a high level of optimism among investors and citizens. Moreover, the Russian income from hydrocarbon exports was substantial. 
            Theoretically, these factors would support the expectation of a normal yield curve at each moment in time. 
            We conducted an analysis of the zero-yield curves (ZYC) for each month and discovered that they exhibited normal 
            behavior in almost every month, further supporting this hypothesis.
        \end{itemize}
        In our future research, we plan to consider exogenous factors and more complicated models to improve our prediction metodology.
        Also, we will consider structural structural breaks and impulse-responses. 

    \references
\end{document} 