\documentclass[aspectratio=169]{beamer}

% \renewcommand{\textSupervisors}    {Supervisors}

\usetheme{vega}

\title{Report on the project "Forecasting the Yield Curve: An Econometric Study"}
\subtitle{Financial Econometrics}
\author{Vsevolod Zaostrovsky, Ivan Cherepakhin, Artemy Sazonov}
\institute{Vega Institute Foundation}
\supervisor{Ivan P. Stankevich}
% \date{August 21 -- 28, 2022}

\usepackage[]{lipsum}
\begin{document}
\maketitle

\begin{frame}{The Data}
    \begin{figure}
    \includegraphics[scale=0.21]{fig/YTMp.pdf}

    \caption{YTM for three different bonds}
    \label{fig:YTMp}
    \end{figure}
\end{frame}

    \begin{frame}{The Yield Curve}
        \begin{figure}
        \includegraphics[scale=0.21]{fig/ZCYp.pdf}
        \caption{The Yield Curves in two different moments of time}
        \label{fig:ZCY}
        \end{figure}
    \end{frame}

    \begin{frame}{The first approach: time series models}
\begin{tabular}{|c c c c c c c|} 
    \hline
    Maturity & autoARIMA & ARIMA(0, 0, 0) & RW & VECM(2) & GARCH \\
    \hline
    3m & $0.0045$ & $0.0047$ & $0.0109$ & $0.0193$ & $0.6115$ \\ 
    \hline
    6m & $0.0039$  & 0.$0041$ & $0.0100$ & $0.0182$ & $0.4658$ \\
    \hline
    9m & $0.0035^{**}$ & $0.0038$ & $0.0095$ & $0.0178$ & $0.5676$ \\
    \hline
    12m & $0.0038^{**}$ & $0.0039$ & $0.0069$ & $0.0194$ & $0.7794$ \\
    \hline
    5y & $0.0052$ & $0.0053$ & $0.0072$ & $0.0182$ & $1.2742$\\
    \hline
    15y & $0.0059$ & $0.0061$ & $0.0076$ & $0.0174$ & $1.9276$ \\ 
    \hline
\end{tabular}
    \end{frame}

    \begin{frame}{The second approach: Nelson--Siegel parametric model}
        The static NS model is defined as follows:
            \begin{equation}\label{eq:NS}
                G(T) = \beta_0 + (\beta_1+\beta_2)\frac{\tau}{T}\left(1-e^{-\frac{T}{\tau}}\right)-\beta_2  e^{-\frac{T}{\tau}},
            \end{equation}
            where $T$ is the time to maturity, $G(T)$ is the yield estimator of the government bonds from the curve basis, 
            and the parameters to be estimated are
            \begin{enumerate}
                \item $\tau$ is the 'typical' time to maturity, 
                \item $\beta_0$ is the long-run of zero-bond yields, 
                \item $\beta_1$ is the mid-run of zero-bond yields, 
                \item $\beta_2$ is the short-run of zero-bond yields.
            \end{enumerate}
    \end{frame}

    \begin{frame}{The second approach: time series models}
    \begin{tabular}{|c | c c c|} 
        \hline
        Coefficient & auto-ARIMA & VAR(1) & RW \\
        \hline
        $\beta_0$ & $53.78356$ & $131.1459$ & $66.3105$ \\ 
        \hline
        $\beta_1$ & $63.31042$ & $143.9235$ & $66.25878$ \\
        \hline
        $\beta_2$ & $133.9688$ & $388.3436$ & $177.1525$ \\
        \hline
        $\tau$ & $1.083687$ & $2.569167$ & $1.328986$ \\
        \hline
    \end{tabular}
\end{frame}

    \begin{frame}{Our conclusions and next steps}
    We concluded that:
    \begin{enumerate}
        \item It is better not to use simple time series 
        models to predict bond returns \ldots
        \item \ldots since the first difference of bonds is martingale relative to the filtering of these models.
        \item Research structural breaks.
    \end{enumerate}
    Our plans on the next research iteration:
    \begin{enumerate}
        \item Try the more complicated modifications of NS model.
        \item Add exogeneous variables.
        \item Research structural breaks.
    \end{enumerate}
    \end{frame}


\end{document}